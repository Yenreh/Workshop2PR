\section{Problema 1.2a: Job Shop con Operarios Limitados}

\subsection{Descripci\'on del Problema}

Esta variaci\'on del JSSP incorpora un n\'umero limitado $k$ de operarios especializados. Cada operaci\'on requiere ejecutarse en una m\'aquina espec\'ifica \textit{y} la supervisi\'on de un operario, quien no puede atender m\'ultiples operaciones simult\'aneamente. Esto limita el paralelismo del sistema a\'un con m\'aquinas disponibles.

\textbf{Elementos adicionales al JSSP cl\'asico:}
\begin{itemize}
    \item $k$ operarios disponibles ($k <$ potencial de tareas simult\'aneas)
    \item Asignaci\'on obligatoria de operario por operaci\'on
    \item Restricci\'on de exclusividad temporal del operario
    \item \textbf{Objetivo dual:} Minimizar makespan y balancear carga entre operarios
\end{itemize}

\subsection{Modelamiento como CSP}

\subsubsection{Par\'ametros y Variables}

\textbf{Par\'ametros:} \texttt{jobs}, \texttt{tasks}, \texttt{k} (operarios), \texttt{d[JOB,TASK]} (duraciones).

\textbf{Variables principales:}
\begin{itemize}
    \item \texttt{s[JOB,TASK]}: Tiempos de inicio de cada operaci\'on
    \item \texttt{o[JOB,TASK]}: Asignaci\'on de operario a cada operaci\'on
    \item \texttt{end}: Makespan (tiempo total)
    \item \texttt{carga[OP]}: Carga de trabajo por operario (para balanceo)
    \item \texttt{used[OP]}: Indicador de operarios utilizados (ruptura de simetr\'ias)
\end{itemize}

\subsubsection{Restricciones Principales}

\textbf{Job Shop cl\'asico:}
\begin{enumerate}
    \item Precedencia: $s[i,j] + d[i,j] \leq s[i,j+1]$ (secuencialidad dentro de cada trabajo)
    \item Capacidad de m\'aquinas: \texttt{disjunctive} por cada tarea (no solapamiento)
    \item Makespan: $s[i,\text{tasks}] + d[i,\text{tasks}] \leq \text{end}$
\end{enumerate}

\textbf{Operarios limitados:}
\begin{enumerate}
    \setcounter{enumi}{3}
    \item No solapamiento: Si $o[i_1,j_1] = o[i_2,j_2]$ entonces intervalos temporales disjuntos
    \item Carga: $\text{carga}[p] = \sum d[i,j] \cdot \mathbb{1}_{o[i,j] = p}$
\end{enumerate}

\textbf{Ruptura de simetr\'ias:}
\begin{enumerate}
    \setcounter{enumi}{5}
    \item Uso consecutivo: $\text{used}[p] \implies \text{used}[p-1]$
    \item Anclaje: $o[1,1] = 1$
    \item Ordenamiento: $\text{carga}[p] \geq \text{carga}[p+1]$
\end{enumerate}

\textbf{Redundante (poda eficiente):}
\begin{enumerate}
    \setcounter{enumi}{8}
    \item \texttt{cumulative}$(S, D, [1,\ldots,1], k)$: A lo sumo $k$ tareas simult\'aneas. Redundante l\'ogicamente pero mejora propagaci\'on dr\'asticamente.
\end{enumerate}

\subsection{Detalles de Implementaci\'on}

\textbf{Aspectos clave:}
\begin{itemize}
    \item Uso de restricciones globales (\texttt{disjunctive}, \texttt{cumulative}) para propagaci\'on eficiente
    \item Formulaci\'on expl\'icita para no-solapamiento de operarios (condicional a asignaci\'on)
    \item Variables derivadas (\texttt{maxload}, \texttt{minload}) para objetivo de balanceo
    \item Funci\'on objetivo: $W \cdot \text{end} + (\text{maxload} - \text{minload})$ con $W = \text{total} + 1$
\end{itemize}

\subsection{Estrategias de B\'usqueda Implementadas}

Se implementaron tres estrategias para evaluar su impacto en calidad de soluci\'on y eficiencia:

\begin{enumerate}
    \item \textbf{Estrategia 1 -- B\'usqueda Libre:} Sin anotaciones expl\'icitas, usa heur\'isticas por defecto del solver. Probada con Gecode, Chuffed y HiGHS para comparar comportamiento entre solvers.
    
    \item \textbf{Estrategia 2 -- \texttt{dom\_w\_deg} para tiempos:} 
    \begin{lstlisting}[language=minizinc]
solve :: seq_search([
  int_search([s[i,j] | ...], dom_w_deg, indomain_min),
  int_search([o[i,j] | ...], first_fail, indomain_min)
]) minimize W * end + (maxload - minload);
\end{lstlisting}
    Decide tiempos con heur\'istica adaptativa (\texttt{dom\_w\_deg}), luego operarios con \texttt{first\_fail}. Probada solo con Gecode.
    
    \item \textbf{Estrategia 3 -- Operarios primero:} Decide operarios antes que tiempos, ambos con \texttt{first\_fail}. Hip\'otesis: agrupar por operario facilita scheduling. Probada solo con Gecode.
\end{enumerate}

\subsection{Configuraci\'on de Pruebas}

Las pruebas fueron ejecutadas mediante el script \texttt{run\_jobshop\_tests.sh} bajo las siguientes condiciones:

\textbf{Par\'ametros del script:}
\begin{itemize}
    \item \textbf{L\'imite de tiempo:} 60,000 ms (60 segundos) por instancia
    \item \textbf{Semilla aleatoria:} 1 (para reproducibilidad)
    \item \textbf{Modo de soluci\'on:} Mejor soluci\'on encontrada (no todas las intermedias)
    \item \textbf{Estad\'isticas habilitadas:} \texttt{-s} (para capturar nodos, propagaciones, fallos)
    \item \textbf{Instancias:} 11 archivos (data00.dzn a data10.dzn)
    \item \textbf{Modelos:} 3 estrategias (jobshop\_search\_1.mzn, jobshop\_search\_2.mzn, jobshop\_search\_3.mzn)
\end{itemize}

\textbf{Solvers utilizados:}
\begin{itemize}
    \item Estrategia 1: Gecode 6.3.0, Chuffed, HiGHS (comparaci\'on multi-solver)
    \item Estrategias 2--3: Gecode 6.3.0 \'unicamente (\texttt{seq\_search} no soportado por otros solvers)
\end{itemize}

\begin{table}[h]
\centering
\small
\begin{tabular}{lccccc}
\toprule
\textbf{Inst.} & \textbf{Jobs} & \textbf{Tasks} & \textbf{k} & \textbf{Ops.} & \textbf{Carga} \\
\midrule
data00 & 4 & 4 & 3 & 16 & 86 \\
data01 & 5 & 5 & 2 & 25 & 108 \\
data02 & 5 & 5 & 3 & 25 & 112 \\
data03 & 5 & 5 & 4 & 25 & 120 \\
data04 & 6 & 4 & 2 & 24 & 118 \\
data05 & 6 & 4 & 3 & 24 & 120 \\
data06 & 6 & 6 & 4 & 36 & 138 \\
data07 & 7 & 5 & 2 & 35 & 162 \\
data08 & 7 & 5 & 3 & 35 & 155 \\
data09 & 8 & 5 & 3 & 40 & 175 \\
data10 & 8 & 6 & 4 & 48 & 187 \\
\bottomrule
\end{tabular}
\caption{Instancias (complejidad creciente, $k/\text{tasks} \in [0.4, 0.8]$)}
\label{tab:instancias}
\end{table}

\subsubsection{Resultados Experimentales}

\paragraph{Estrategia 1 -- Comparaci\'on Multi-Solver}

\begin{table}[h]
\centering
\tiny
\begin{tabular}{lrr|rr|rr}
\toprule
& \multicolumn{2}{c}{\textbf{Gecode}} & \multicolumn{2}{c}{\textbf{Chuffed}} & \multicolumn{2}{c}{\textbf{HiGHS}} \\
\textbf{Inst.} & \textbf{Mks} & \textbf{Desb} & \textbf{Mks} & \textbf{Desb} & \textbf{Mks} & \textbf{Desb} \\
\midrule
data00 & 86 & 86 & 32 & 1 & 32 & 1 \\
data01 & 108 & 108 & 58 & 2 & 55 & 0 \\
data02 & 112 & 112 & 51 & 1 & 44 & 1 \\
data03 & 120 & 120 & 45 & 16 & 45 & 0 \\
data04 & 118 & 118 & 110 & 4 & 64 & 4 \\
data05 & 120 & 120 & 61 & 7 & 43 & 0 \\
data06 & 138 & 138 & 73 & 38 & 50 & 1 \\
data07 & 162 & 162 & 140 & 0 & 88 & 0 \\
data08 & 155 & 155 & 152 & 78 & 64 & 2 \\
data09 & 175 & 175 & 174 & 98 & 70 & 1 \\
data10 & 187 & 187 & 187 & 1 & 69 & 14 \\
\midrule
\textbf{Prom.} & 134.6 & 134.6 & 98.5 & 22.4 & 56.7 & 2.2 \\
\bottomrule
\end{tabular}
\caption{Estrategia 1 (b\'usqueda libre) con 3 solvers}
\label{tab:est1_solvers}
\end{table}


\begin{itemize}
    \item \textbf{Gecode:} 100\% triviales, \textbf{todas las instancias alcanzaron el l\'imite de 60s} sin encontrar soluciones no triviales
    \item \textbf{Chuffed:} 27\% cr\'iticos, \textbf{todas las instancias alcanzaron timeout} pero con mejores soluciones intermedias
    \item \textbf{HiGHS:} 0\% problem\'aticos, 45\% perfectos, \textbf{ninguna instancia alcanz\'o timeout} (resoluci\'on completa)
\end{itemize}
\textbf{HiGHS claramente superior.}

\paragraph{Estrategias 2 y 3 -- Solo Gecode}

\begin{table}[h]
\centering
\resizebox{\textwidth}{!}{%
\begin{tabular}{lrrrr|rrrr|rrrr}
\toprule
& \multicolumn{4}{c}{\textbf{Est. 1 (libre)}} & \multicolumn{4}{c}{\textbf{Est. 2 (dom\_w\_deg)}} & \multicolumn{4}{c}{\textbf{Est. 3 (op primero)}} \\
\textbf{Inst.} & \textbf{Mks} & \textbf{Dsb} & \textbf{Nod(M)} & \textbf{Prop(M)} & \textbf{Mks} & \textbf{Dsb} & \textbf{Nod(M)} & \textbf{Prop(M)} & \textbf{Mks} & \textbf{Dsb} & \textbf{Nod(M)} & \textbf{Prop(M)} \\
\midrule
data00 & 86 & 86 & 8.7 & 929 & 39 & 1 & 11.0 & 1179 & 86 & 86 & 10.7 & 1147 \\
data01 & 108 & 108 & 9.4 & 935 & 57 & 4 & 4.9 & 348 & 108 & 108 & 15.2 & 1026 \\
data02 & 112 & 112 & 9.1 & 911 & 54 & 1 & 10.0 & 1205 & 112 & 112 & 11.3 & 1086 \\
data03 & 120 & 120 & 9.0 & 802 & 45 & 0 & 0.004 & 1.4 & 120 & 120 & 12.8 & 1098 \\
data04 & 118 & 118 & 8.2 & 842 & 62 & 0 & 2.9 & 487 & 118 & 118 & 13.6 & 996 \\
data05 & 120 & 120 & 8.5 & 858 & 43 & 0 & 0.4 & 52 & 120 & 120 & 16.7 & 1153 \\
data06 & 138 & 138 & 8.7 & 784 & 65 & 1 & 8.6 & 952 & 138 & 138 & 15.5 & 1067 \\
data07 & 162 & 162 & 7.9 & 664 & 86 & 6 & 2.5 & 363 & 162 & 162 & 9.8 & 942 \\
data08 & 155 & 155 & 8.5 & 794 & 66 & 1 & 6.5 & 1050 & 155 & 155 & 8.7 & 958 \\
data09 & 175 & 175 & 7.6 & 710 & 71 & 1 & 5.9 & 996 & 175 & 175 & 8.1 & 818 \\
data10 & 187 & 187 & 7.5 & 751 & 84 & 1 & 6.2 & 858 & 187 & 187 & 8.7 & 622 \\
\midrule
\textbf{Prom.} & 134.6 & 134.6 & 8.4 & 816 & 61.1 & 1.5 & 5.4 & 772 & 134.6 & 134.6 & 11.9 & 992 \\
\bottomrule
\end{tabular}%
}
\caption{Comparaci\'on de estrategias en Gecode 6.3.0}
\label{tab:gecode_estrategias}
\end{table}


\begin{itemize}
    \item \textbf{Est.1 vs Est.2:} \texttt{dom\_w\_deg} reduce makespan 55\% (134.6 $\rightarrow$ 61.1) y desbalance 99\% (134.6 $\rightarrow$ 1.5). \textbf{Est.1 y Est.3 alcanzaron timeout en todas las instancias}, mientras que \textbf{Est.2 tambi\'en alcanz\'o timeout universal} pero encontrando soluciones dr\'asticamente mejores.
    \item \textbf{Est.1 vs Est.3:} Calidad id\'entica (ambas triviales por timeout), pero Est.3 usa 42\% m\'as nodos y 22\% m\'as propagaciones antes de agotar tiempo.
    \item \textbf{Est.2 eficiencia:} Mejor calidad con menos nodos promedio que Est.1 (5.4M vs 8.4M), pero alta variabilidad. A pesar del timeout, la heur\'istica adaptativa explora mejor el espacio de b\'usqueda.
    \item \textbf{data03 notable:} Est.2 resuelve con solo 4K nodos (balanceo perfecto), vs 9M en Est.1. \'Unica instancia sin timeout en Est.2.
\end{itemize}

\subsection{An\'alisis Comparativo}

\subsubsection{Hallazgo 1: El Solver Importa Tanto Como la Estrategia}

Comparando resultados de Estrategia 1 (libre) entre solvers:

\begin{itemize}
    \item \textbf{HiGHS:} Makespan 56.7, desbalance 2.2, ~2.5K nodos
    \item \textbf{Gecode Est.2 (con anotaciones):} Makespan 61.1, desbalance 1.5, ~5.4M nodos
    \item \textbf{Chuffed:} Makespan 98.5, desbalance 22.4, ~1.5M nodos
    \item \textbf{Gecode Est.1:} Makespan 134.6, desbalance 134.6, ~8.4M nodos
\end{itemize}

\textbf{Conclusi\'on:} HiGHS sin anotaciones supera a Gecode con \texttt{dom\_w\_deg}, demostrando que heur\'isticas internas bien dise\~nadas eliminan necesidad de anotaciones expl\'icitas. Cabe destacar, que los resultados triviales de Gecode Est.1 se deben al agotamiento del l\'imite de tiempo (60s) sin encontrar soluciones mejores, mientras que HiGHS resuelve \'optimamente sin alcanzar el timeout.

\subsubsection{Hallazgo 2: Impacto de Estrategias en Gecode}

Comparando las 3 estrategias implementadas (ver Tabla \ref{tab:gecode_estrategias}):


\begin{itemize}
    \item \textbf{Est.1:} Makespan prom. 134.6, desbalance 134.6 (100\% triviales), 8.4M nodos, \textbf{timeout en todas las instancias}
    \item \textbf{Est.2:} Makespan prom. 61.1 (55\% mejor), desbalance 1.5 (99\% mejor), 5.4M nodos, \textbf{timeout en 10/11 instancias} (solo data03 termin\'o antes)
    \item \textbf{Est.3:} Makespan prom. 134.6 (id\'entico a Est.1), 11.9M nodos (42\% peor eficiencia), \textbf{timeout en todas las instancias}
    \item \textbf{Est.2 eficiencia variable:} data03 con 4K nodos (\'unica sin timeout) vs data00 con 11M nodos (alta dependencia de estructura)
    \item \textbf{Timeout universal en Gecode:} Las 3 estrategias agotan el l\'imite de 60s en la mayor\'ia de instancias, pero Est.2 encuentra mejores soluciones intermedias gracias a \texttt{dom\_w\_deg}
\end{itemize}

\subsection{Conclusiones}

\begin{enumerate}
    \item \textbf{HiGHS es el mejor solver para este problema:} Sin anotaciones de b\'usqueda, logra makespan promedio 56.7 (58\% mejor que Gecode sin anotaciones, 7\% mejor que Gecode con \texttt{dom\_w\_deg}), desbalance m\'aximo 14, y solo ~2.5K nodos. \textbf{Ninguna instancia alcanz\'o el l\'imite de 60s,} indicando resoluci\'on completa. Sus heur\'isticas internas son superiores.
    
    \item \textbf{Las heur\'isticas por defecto var\'ian dr\'asticamente:} Gecode produce soluciones triviales (100\% casos) \textbf{agotando el timeout en todas las instancias}, Chuffed tiene 27\% casos cr\'iticos \textbf{tambi\'en con timeout universal}, HiGHS 0\% casos problem\'aticos \textbf{sin alcanzar timeout}. La elecci\'on del solver es tan importante como la estrategia de b\'usqueda.
    
    \item \textbf{Anotaciones expl\'icitas necesarias en Gecode:} \texttt{dom\_w\_deg} mejora Gecode 55\% (makespan 134.6 $\rightarrow$ 61.1), pero a\'un queda 8\% por debajo de HiGHS sin anotaciones. \textbf{A pesar de alcanzar timeout en 10/11 instancias,} las heur\'isticas adaptativas encuentran mejores soluciones intermedias, aunque no compensan totalmente deficiencias del solver base.
    
    \item \textbf{Decidir operarios primero es contraproducente:} Estrategia 3 produce resultados id\'enticos a Estrategia 1 pero con 42\% m\'as nodos antes de agotar tiempo. El orden temporal debe decidirse antes que asignaciones de recursos.
    
    \item \textbf{Limitaci\'on de portabilidad:} \texttt{seq\_search} solo funciona en Gecode, imposibilitando aplicar Estrategia 2 a HiGHS para verificar si mejora a\'un m\'as sus ya excelentes resultados.
\end{enumerate}

\textbf{Recomendaci\'on:} Usar HiGHS sin anotaciones para este tipo de problemas. Si se requiere Gecode, aplicar Estrategia 2 (\texttt{dom\_w\_deg}).
