\section{Problema 1.2a: Job Shop con Operarios Limitados}

\subsection{Descripci\'on del Problema}

En esta variaci\'on del Job Shop Scheduling Problem, se introduce la restricci\'on de que hay un n\'umero limitado de operarios especializados disponibles. Aunque cada operaci\'on requiere ejecutarse en una m\'aquina espec\'ifica, tambi\'en requiere la presencia de un operario para supervisarla u operarla. Como hay menos operarios que m\'aquinas, no todas las m\'aquinas pueden funcionar simult\'aneamente, lo que a\~nade una nueva dimensi\'on de complejidad al problema.

\textbf{Contexto industrial:} En entornos de producci\'on modernos, aunque las m\'aquinas puedan ser numerosas y especializadas, el personal calificado para operarlas es un recurso limitado y costoso. La planificaci\'on eficiente debe considerar tanto la disponibilidad de m\'aquinas como de operarios, buscando adem\'as balancear la carga de trabajo para evitar cuellos de botella y fatiga.

\textbf{Elementos del problema:}
\begin{itemize}
    \item Un conjunto de \textit{jobs} con sus respectivas secuencias de \textit{tasks}
    \item Cada operaci\'on tiene duraci\'on fija y m\'aquina asignada
    \item Restricciones de precedencia dentro de cada trabajo
    \item Cada m\'aquina procesa una operaci\'on a la vez
    \item \textbf{Nuevo:} Solo hay $k$ operarios disponibles (donde $k <$ n\'umero de tareas concurrentes potenciales)
    \item \textbf{Nuevo:} Cada operaci\'on debe ser asignada a exactamente un operario
    \item \textbf{Nuevo:} Un operario no puede estar en dos operaciones simult\'aneas
    \item \textbf{Objetivos duales:}
    \begin{enumerate}
        \item Minimizar el makespan (tiempo total)
        \item Balancear la carga de trabajo entre operarios
    \end{enumerate}
\end{itemize}

\subsection{Modelamiento como CSP}

\subsubsection{Par\'ametros del Modelo}

Los par\'ametros definen la instancia del problema:

\begin{itemize}
    \item \textbf{jobs} (\texttt{int}): N\'umero de trabajos a planificar
    \item \textbf{tasks} (\texttt{int}): N\'umero de operaciones por trabajo (asumiendo estructura uniforme)
    \item \textbf{k} (\texttt{int}): N\'umero de operarios disponibles
    \item \textbf{d} (\texttt{array[JOB, TASK] of int}): Matriz de duraciones, donde $d[i,j]$ es la duraci\'on de la tarea $j$ del trabajo $i$
\end{itemize}

Adicionalmente se definen conjuntos para facilitar la iteraci\'on:
\begin{itemize}
    \item \textbf{JOB} = $\{1, \ldots, \text{jobs}\}$
    \item \textbf{TASK} = $\{1, \ldots, \text{tasks}\}$
    \item \textbf{OP} = $\{1, \ldots, k\}$: Conjunto de operarios
\end{itemize}

\textbf{Justificaci\'on:} Esta estructura permite modelar problemas de tama\~no variable manteniendo la claridad del modelo. La suposici\'on de estructura uniforme (mismo n\'umero de tareas por trabajo) simplifica el modelo sin perder generalidad significativa.

\subsubsection{Variables de Decisi\'on}

El modelo define las siguientes variables de decisi\'on:

\begin{enumerate}
    \item \textbf{s} (\texttt{array[JOB, TASK] of var 0..total}): 
    \begin{itemize}
        \item $s[i,j]$ representa el tiempo de inicio de la tarea $j$ del trabajo $i$
        \item Dominio: $[0, \text{total}]$ donde $\text{total} = \sum_{i,j} d[i,j]$ (cota superior trivial)
        \item Esta es la variable principal que determina el cronograma
    \end{itemize}
    
    \item \textbf{o} (\texttt{array[JOB, TASK] of var OP}): 
    \begin{itemize}
        \item $o[i,j]$ representa qu\'e operario es asignado a la tarea $j$ del trabajo $i$
        \item Dominio: $\{1, \ldots, k\}$ (identificadores de operarios)
        \item Esta variable conecta las restricciones de capacidad de operarios con el cronograma
    \end{itemize}
    
    \item \textbf{end} (\texttt{var 0..total}):
    \begin{itemize}
        \item Representa el tiempo de finalizaci\'on del \'ultimo trabajo (makespan)
        \item Variable objetivo principal
    \end{itemize}
    
    \item \textbf{used} (\texttt{array[OP] of var bool}):
    \begin{itemize}
        \item $\text{used}[p]$ es verdadero si y solo si el operario $p$ es asignado a al menos una tarea
        \item Variable auxiliar para romper simetr\'ias
    \end{itemize}
    
    \item \textbf{carga} (\texttt{array[OP] of var 0..total}):
    \begin{itemize}
        \item $\text{carga}[p]$ representa la carga total de trabajo asignada al operario $p$
        \item Suma de duraciones de todas las tareas asignadas a ese operario
        \item Variable derivada para el objetivo de balanceo
    \end{itemize}
\end{enumerate}

\textbf{Justificaci\'on del modelamiento:} La elecci\'on de representar tanto tiempos de inicio como asignaciones de operarios como variables permite al solver explorar ambas dimensiones simult\'aneamente, facilitando la propagaci\'on de restricciones. Las variables auxiliares (\texttt{used}, \texttt{carga}) son t\'ecnicamente redundantes pero esenciales para expresar restricciones de simetr\'ia y objetivos de balanceo de forma clara y eficiente.

\subsubsection{Restricciones}

\paragraph{Restricciones del Job Shop Cl\'asico}

\textbf{1. Precedencia dentro de cada trabajo:}
\begin{equation}
\forall i \in \text{JOB}, \forall j \in \{1, \ldots, \text{tasks}-1\}: s[i,j] + d[i,j] \leq s[i,j+1]
\end{equation}

\textit{Interpretaci\'on:} La tarea $j+1$ del trabajo $i$ no puede comenzar hasta que la tarea $j$ haya terminado. Esto captura la dependencia secuencial de operaciones dentro de un trabajo.

\textbf{2. Capacidad de m\'aquinas (una operaci\'on por tarea a la vez):}
\begin{equation}
\forall j \in \text{TASK}: \text{disjunctive}([s[i,j] \mid i \in \text{JOB}], [d[i,j] \mid i \in \text{JOB}])
\end{equation}

\textit{Interpretaci\'on:} Para cada \'indice de tarea $j$ (que corresponde a una m\'aquina en el modelo cl\'asico), todas las operaciones que usan esa m\'aquina deben ejecutarse sin solapamiento temporal. La restricci\'on global \texttt{disjunctive} asegura que los intervalos $[s[i,j], s[i,j] + d[i,j])$ sean disjuntos.

\textbf{3. Definici\'on del makespan:}
\begin{equation}
\forall i \in \text{JOB}: s[i, \text{tasks}] + d[i, \text{tasks}] \leq \text{end}
\end{equation}

\textit{Interpretaci\'on:} El makespan debe ser al menos el tiempo de finalizaci\'on del \'ultimo trabajo completado.

\paragraph{Restricciones de Operarios}

\textbf{4. No solapamiento de operarios:}
\begin{equation}
\begin{aligned}
\forall (i_1, j_1), (i_2, j_2) \in \text{JOB} \times \text{TASK} \text{ con } (i_1,j_1) < (i_2,j_2): \\
o[i_1,j_1] = o[i_2,j_2] \implies \\
(s[i_1,j_1] + d[i_1,j_1] \leq s[i_2,j_2] \lor s[i_2,j_2] + d[i_2,j_2] \leq s[i_1,j_1])
\end{aligned}
\end{equation}

\textit{Interpretaci\'on:} Si dos tareas son asignadas al mismo operario, sus intervalos temporales no pueden solaparse. Esta es la restricci\'on fundamental que limita el paralelismo en el sistema.

\textbf{Nota t\'ecnica:} La condici\'on $(i_1,j_1) < (i_2,j_2)$ (orden lexicogr\'afico) evita comparar cada par dos veces y la auto-comparaci\'on.

\textbf{5. Definici\'on de carga por operario:}
\begin{equation}
\forall p \in \text{OP}: \text{carga}[p] = \sum_{i \in \text{JOB}, j \in \text{TASK}} d[i,j] \cdot \mathbb{1}_{o[i,j] = p}
\end{equation}

donde $\mathbb{1}_{o[i,j] = p}$ es 1 si $o[i,j] = p$, y 0 en caso contrario.

\textit{Interpretaci\'on:} La carga de cada operario es la suma de duraciones de todas las tareas que se le asignan.

\paragraph{Ruptura de Simetr\'ias}

\textbf{6. Uso consecutivo de operarios:}
\begin{equation}
\begin{aligned}
&\forall p \in \text{OP}: \text{used}[p] \iff \exists (i,j) \in \text{JOB} \times \text{TASK}: o[i,j] = p \\
&\forall p \in \{2, \ldots, k\}: \text{used}[p] \implies \text{used}[p-1]
\end{aligned}
\end{equation}

\textit{Interpretaci\'on:} Forzamos que los operarios se usen consecutivamente (1, 2, 3, ...), eliminando simetr\'ias por permutaci\'on de operarios id\'enticos.

\textbf{7. Anclaje del primer operario:}
\begin{equation}
o[1,1] = 1
\end{equation}

\textit{Interpretaci\'on:} La primera tarea del primer trabajo siempre usa el operario 1, eliminando rotaciones del conjunto de operarios.

\textbf{8. Ordenamiento de cargas:}
\begin{equation}
\forall p \in \{1, \ldots, k-1\}: \text{carga}[p] \geq \text{carga}[p+1]
\end{equation}

\textit{Interpretaci\'on:} Las cargas se ordenan de mayor a menor, eliminando permutaciones de operarios con cargas id\'enticas. Esta restricci\'on es compatible con el objetivo de balanceo.

\paragraph{Restricci\'on Redundante}

\textbf{9. Restricci\'on cumulative (poda de dominios):}
\begin{equation}
\text{cumulative}(S_{\text{all}}, D_{\text{all}}, R_{\text{all}}, k)
\end{equation}

donde:
\begin{itemize}
    \item $S_{\text{all}} = [s[i,j] \mid i \in \text{JOB}, j \in \text{TASK}]$: vector de tiempos de inicio
    \item $D_{\text{all}} = [d[i,j] \mid i \in \text{JOB}, j \in \text{TASK}]$: vector de duraciones
    \item $R_{\text{all}} = [1 \mid i \in \text{JOB}, j \in \text{TASK}]$: cada tarea requiere 1 recurso
    \item $k$: capacidad m\'axima del recurso
\end{itemize}

\textit{Interpretaci\'on:} En cualquier momento, a lo sumo $k$ tareas pueden estar ejecut\'andose simult\'aneamente (porque solo hay $k$ operarios). Esta restricci\'on es \textbf{redundante} porque se deriva l\'ogicamente de la restricci\'on 4, pero la restricci\'on global \texttt{cumulative} implementa algoritmos de propagaci\'on especializados que podan el espacio de b\'usqueda m\'as eficientemente.

\textbf{¿Por qu\'e es \'util esta redundancia?} 
\begin{itemize}
    \item La restricci\'on 4 propaga cuando dos tareas comparten operario
    \item La restricci\'on \texttt{cumulative} propaga globalmente sobre todas las tareas simult\'aneas
    \item En la pr\'actica, reduce dr\'asticamente los nodos explorados
\end{itemize}

\subsection{Detalles de Implementaci\'on en MiniZinc}

\subsubsection{Aspectos Relevantes de la Implementaci\'on}

\textbf{1. Uso de restricciones globales:}
\begin{lstlisting}[language=minizinc]
constraint forall(j in TASK) (
  disjunctive([s[i,j] | i in JOB], [d[i,j] | i in JOB])
);
\end{lstlisting}

La restricci\'on global \texttt{disjunctive} es mucho m\'as eficiente que descomponer manualmente en disyunciones. Implementa algoritmos especializados de propagaci\'on para scheduling.

\textbf{2. Restricciones de no-solapamiento de operarios:}

Se opt\'o por una formulaci\'on expl\'icita con implicaci\'on en lugar de usar \texttt{disjunctive} para operarios porque:
\begin{itemize}
    \item La asignaci\'on de operarios es una variable de decisi\'on
    \item El no-solapamiento es condicional (solo si comparten operario)
    \item Permite mayor control sobre la propagaci\'on
\end{itemize}

\textbf{3. Variables auxiliares para balanceo:}

Las variables \texttt{maxload} y \texttt{minload} se definen como:
\begin{lstlisting}[language=minizinc]
var 0..total: maxload = max(p in OP)(carga[p]);
var 0..total: minload = min(p in OP)(carga[p]);
\end{lstlisting}

Estas son variables derivadas pero el solver las trata eficientemente gracias a las optimizaciones de MiniZinc.

\textbf{4. Cota superior opcional:}

El modelo incluye un par\'ametro \texttt{ub\_end} que permite especificar una cota superior conocida para el makespan, facilitando la prueba de instancias espec\'ificas o la comparaci\'on con soluciones conocidas.

\subsection{Estrategias de B\'usqueda}

Se implementaron tres estrategias de b\'usqueda diferentes para evaluar su impacto en el rendimiento:

\subsubsection{Estrategia 1: B\'usqueda Libre}

\begin{lstlisting}[language=minizinc]
solve minimize W * end + (maxload - minload);
\end{lstlisting}

\textbf{Caracter\'isticas:}
\begin{itemize}
    \item No se especifica una anotaci\'on de b\'usqueda expl\'icita
    \item El solver (Gecode) usa sus heur\'isticas por defecto
    \item T\'ipicamente explora variables en orden de declaraci\'on con seleccci\'on de valor por defecto
\end{itemize}

\textbf{Funci\'on objetivo:} $W \cdot \text{end} + (\text{maxload} - \text{minload})$

donde $W = \text{total} + 1$ es un peso grande que prioriza makespan sobre balanceo.

\textbf{Ventajas esperadas:} Simplicidad, sin necesidad de ajustar par\'ametros.

\textbf{Desventajas esperadas:} Puede explorar variables en orden subóptimo, llevando a mayor n\'umero de nodos.

\subsubsection{Estrategia 2: B\'usqueda Secuencial con dom\_w\_deg}

\begin{lstlisting}[language=minizinc]
solve
:: seq_search([
     int_search([s[i,j] | i in JOB, j in TASK], 
                dom_w_deg, indomain_min),
     int_search([o[i,j] | i in JOB, j in TASK], 
                first_fail, indomain_min)
   ])
minimize W * end + (maxload - minload);
\end{lstlisting}

\textbf{Caracter\'isticas:}
\begin{itemize}
    \item \textbf{Fase 1:} Decide tiempos de inicio (\texttt{s}) usando \texttt{dom\_w\_deg}
    \begin{itemize}
        \item \texttt{dom\_w\_deg}: Selecciona la variable con mayor ratio de peso-de-restricciones / tama\~no-de-dominio
        \item ``Peso'' se incrementa cuando una restricci\'on falla
        \item Prioriza variables que han causado conflictos recientemente
        \item \texttt{indomain\_min}: Prueba el valor m\'inimo del dominio primero
    \end{itemize}
    \item \textbf{Fase 2:} Decide asignaciones de operarios (\texttt{o}) usando \texttt{first\_fail}
    \begin{itemize}
        \item \texttt{first\_fail}: Selecciona la variable con el dominio m\'as peque\~no
        \item Ataca primero las decisiones m\'as restringidas
    \end{itemize}
\end{itemize}

\textbf{Justificaci\'on:}
\begin{itemize}
    \item Los tiempos de inicio son m\'as cr\'iticos para el makespan
    \item \texttt{dom\_w\_deg} se adapta din\'amicamente a la estructura del problema
    \item Las asignaciones de operarios pueden propagarse despu\'es de fijar tiempos
\end{itemize}

\textbf{Ventajas esperadas:} Mejor adaptaci\'on a conflictos, exploraci\'on m\'as guiada.

\textbf{Desventajas esperadas:} Mayor overhead computacional por heur\'istica din\'amica.

\subsubsection{Estrategia 3: Operarios Primero}

\begin{lstlisting}[language=minizinc]
solve
:: seq_search([
     int_search([o[i,j] | i in JOB, j in TASK], 
                first_fail, indomain_min),
     int_search([s[i,j] | i in JOB, j in TASK], 
                first_fail, indomain_min)
   ])
minimize W * end + (maxload - minload);
\end{lstlisting}

\textbf{Caracter\'isticas:}
\begin{itemize}
    \item \textbf{Fase 1:} Decide primero todas las asignaciones de operarios
    \item \textbf{Fase 2:} Luego decide tiempos de inicio
    \item Ambas fases usan \texttt{first\_fail}
\end{itemize}

\textbf{Justificaci\'on:}
\begin{itemize}
    \item Agrupar tareas por operario antes de planificar tiempos
    \item Puede facilitar la detecci\'on temprana de infactibilidades relacionadas con operarios
    \item Una vez fijados los operarios, el problema se descompone en subproblemas de scheduling
\end{itemize}

\textbf{Ventajas esperadas:} Buena descomposici\'on del problema, posible poda temprana.

\textbf{Desventajas esperadas:} Puede fijar asignaciones de operarios prematuramente sin considerar consecuencias temporales.

\subsection{Pruebas Realizadas}

\subsubsection{Configuraci\'on de las Pruebas}

Las pruebas se realizaron utilizando el script \texttt{run\_jobshop\_tests.sh} con los siguientes par\'ametros:

\begin{itemize}
    \item \textbf{Solver:} Gecode 6.3.0
    \item \textbf{L\'imite de tiempo:} 60,000 ms (60 segundos) por instancia
    \item \textbf{Semilla aleatoria:} 1 (para reproducibilidad)
    \item \textbf{Modelos evaluados:} 3 estrategias de b\'usqueda
    \item \textbf{Instancias:} 11 archivos de datos (data00.dzn a data10.dzn)
\end{itemize}

\textbf{Justificaci\'on del l\'imite de tiempo:} Las instancias de mayor tama\~no presentaban tiempos de ejecuci\'on excesivamente largos (superiores a 8 minutos con m\'ultiples hilos) sin convergencia. El l\'imite de 60 segundos permite evaluar el rendimiento relativo de las estrategias en un tiempo razonable, priorizando la exploraci\'on inicial del espacio de b\'usqueda.

\subsubsection{Descripci\'on de las Instancias}

Todas las instancias siguen el formato:
\begin{lstlisting}[language=minizinc]
jobs = <numero>;
tasks = <numero>;
k = <numero>;
d = [| <matriz de duraciones> |];
\end{lstlisting}

\textbf{Caracter\'isticas de las instancias:}

\begin{table}[h]
\centering
\begin{tabular}{lccccc}
\toprule
\textbf{Instancia} & \textbf{Jobs} & \textbf{Tasks} & \textbf{Operarios (k)} & \textbf{Operaciones Totales} & \textbf{Carga Total} \\
\midrule
data00 & 4 & 4 & 3 & 16 & 86 \\
data01 & 5 & 5 & 2 & 25 & 108 \\
data02 & 5 & 5 & 3 & 25 & 112 \\
data03 & 5 & 5 & 4 & 25 & 120 \\
data04 & 6 & 4 & 2 & 24 & 118 \\
data05 & 6 & 4 & 3 & 24 & 120 \\
data06 & 6 & 6 & 4 & 36 & 138 \\
data07 & 7 & 5 & 2 & 35 & 162 \\
data08 & 7 & 5 & 3 & 35 & 155 \\
data09 & 8 & 5 & 3 & 40 & 175 \\
data10 & 8 & 6 & 4 & 48 & 187 \\
\bottomrule
\end{tabular}
\caption{Caracter\'isticas de las instancias de prueba}
\label{tab:instancias}
\end{table}

\textbf{Observaciones:}
\begin{itemize}
    \item Las instancias aumentan progresivamente en complejidad
    \item La relaci\'on operarios/tareas var\'ia (desde $k/\text{tasks} = 0.4$ hasta $0.8$)
    \item La carga total aumenta de 86 a 187
\end{itemize}

\subsubsection{Resultados Experimentales}

\paragraph{Estrategia 1: B\'usqueda Libre}

\begin{table}[h]
\centering
\small
\begin{tabular}{lrrrrr}
\toprule
\textbf{Instancia} & \textbf{Makespan} & \textbf{Desbalance} & \textbf{Nodos} & \textbf{Fallos} & \textbf{Propagaciones} \\
\midrule
data00 & 86 & 86 & 8,656,470 & 4,328,217 & 928,732,226 \\
data01 & 108 & 108 & 9,380,158 & 4,690,062 & 934,887,110 \\
data02 & 112 & 112 & 9,141,499 & 4,570,731 & 910,658,768 \\
data03 & 120 & 120 & 8,977,618 & 4,488,791 & 801,706,349 \\
data04 & 118 & 118 & 8,163,317 & 4,081,638 & 841,537,470 \\
data05 & 120 & 120 & 8,483,181 & 4,241,571 & 858,491,031 \\
data06 & 138 & 138 & 8,708,307 & 4,354,133 & 783,575,748 \\
data07 & 162 & 162 & 7,897,032 & 3,948,492 & 663,627,990 \\
data08 & 155 & 155 & 8,461,777 & 4,230,865 & 793,876,685 \\
data09 & 175 & 175 & 7,587,331 & 3,793,638 & 710,109,798 \\
data10 & 187 & 187 & 7,500,955 & 3,750,452 & 750,849,628 \\
\bottomrule
\end{tabular}
\caption{Resultados de Estrategia 1 (B\'usqueda Libre)}
\label{tab:resultados_s1}
\end{table}

\textbf{Observaciones cr\'iticas:}
\begin{itemize}
    \item \textbf{Desbalance m\'aximo:} En todas las instancias, el desbalance es igual al makespan
    \item \textbf{Interpretaci\'on:} Esto indica que \textbf{todos los operarios excepto uno tienen carga cero}
    \item En la columna de carga (no mostrada) se observa: $[\text{carga\_total}, 0, 0, \ldots]$
    \item \textbf{Causa:} La estrategia por defecto no considera el objetivo de balanceo efectivamente
    \item \textbf{N\'umero de nodos:} Muy alto (~8-9 millones en instancias peque\~nas)
    \item \textbf{Propagaciones:} Entre 600 millones y 900 millones
\end{itemize}

\textbf{Conclusi\'on:} Esta estrategia encuentra soluciones triviales donde un solo operario hace todo el trabajo. Aunque t\'ecnicamente v\'alidas, estas soluciones no son pr\'acticas ni deseables.

\paragraph{Estrategia 2: B\'usqueda Secuencial con dom\_w\_deg}

\begin{table}[h]
\centering
\small
\begin{tabular}{lrrrrr}
\toprule
\textbf{Instancia} & \textbf{Makespan} & \textbf{Desbalance} & \textbf{Nodos} & \textbf{Fallos} & \textbf{Propagaciones} \\
\midrule
data00 & 39 & 1 & 11,048,555 & 5,524,261 & 1,178,703,001 \\
data01 & 57 & 4 & 4,946,586 & 2,473,281 & 347,713,759 \\
data02 & 54 & 1 & 9,969,532 & 4,984,745 & 1,204,675,845 \\
data03 & 45 & 0 & 4,061 & 1,955 & 1,410,504 \\
data04 & 62 & 0 & 2,868,894 & 1,434,410 & 486,862,749 \\
data05 & 43 & 0 & 396,851 & 198,299 & 52,210,307 \\
data06 & 65 & 1 & 8,615,337 & 4,307,617 & 952,185,260 \\
data07 & 86 & 6 & 2,513,049 & 1,256,462 & 363,079,829 \\
data08 & 66 & 1 & 6,460,261 & 3,230,108 & 1,050,313,215 \\
data09 & 71 & 1 & 5,859,083 & 2,929,506 & 996,370,298 \\
data10 & 84 & 1 & 6,196,054 & 3,097,958 & 857,916,673 \\
\bottomrule
\end{tabular}
\caption{Resultados de Estrategia 2 (dom\_w\_deg para tiempos)}
\label{tab:resultados_s2}
\end{table}

\textbf{Observaciones importantes:}
\begin{itemize}
    \item \textbf{Makespan dram\'aticamente reducido:} Comparado con Estrategia 1
    \begin{itemize}
        \item data00: 86 $\rightarrow$ 39 (54.7\% reducci\'on)
        \item data01: 108 $\rightarrow$ 57 (47.2\% reducci\'on)
        \item data03: 120 $\rightarrow$ 45 (62.5\% reducci\'on)
    \end{itemize}
    \item \textbf{Balanceo significativo:} Desbalance entre 0 y 6
    \begin{itemize}
        \item data03, data04, data05: Desbalance = 0 (balanceo perfecto)
        \item Mayor\'ia: Desbalance $\leq$ 1 (casi perfecto)
    \end{itemize}
    \item \textbf{Nodos explorados:} Variable, pero con casos de eficiencia extrema
    \begin{itemize}
        \item data03: Solo 4,061 nodos (vs 8.9M en Estrategia 1)
        \item data05: 396,851 nodos (vs 8.4M)
    \end{itemize}
    \item \textbf{Propagaciones:} A\'un muy altas, pero eficiencia por nodo mejorada
\end{itemize}

\textbf{Conclusi\'on:} Esta estrategia es \textbf{significativamente superior}, logrando soluciones de alta calidad con balanceo efectivo y makespans m\'inimos. La heur\'istica \texttt{dom\_w\_deg} demuestra su valor adaptativo.

\paragraph{Estrategia 3: Operarios Primero}

\begin{table}[h]
\centering
\small
\begin{tabular}{lrrrrr}
\toprule
\textbf{Instancia} & \textbf{Makespan} & \textbf{Desbalance} & \textbf{Nodos} & \textbf{Fallos} & \textbf{Propagaciones} \\
\midrule
data00 & 86 & 86 & 10,717,053 & 5,358,504 & 1,146,528,186 \\
data01 & 108 & 108 & 15,223,333 & 7,611,644 & 1,026,296,393 \\
data02 & 112 & 112 & 11,312,284 & 5,656,120 & 1,085,806,822 \\
data03 & 120 & 120 & 12,804,492 & 6,402,224 & 1,098,461,046 \\
data04 & 118 & 118 & 13,626,036 & 6,812,990 & 995,898,055 \\
data05 & 120 & 120 & 16,699,548 & 8,349,745 & 1,152,613,520 \\
data06 & 138 & 138 & 15,456,023 & 7,727,984 & 1,066,657,338 \\
data07 & 162 & 162 & 9,797,646 & 4,898,789 & 942,471,122 \\
data08 & 155 & 155 & 8,666,460 & 4,333,197 & 958,498,638 \\
data09 & 175 & 175 & 8,088,673 & 4,044,299 & 817,757,449 \\
data10 & 187 & 187 & 8,660,372 & 4,330,147 & 621,775,327 \\
\bottomrule
\end{tabular}
\caption{Resultados de Estrategia 3 (Operarios Primero)}
\label{tab:resultados_s3}
\end{table}

\textbf{Observaciones:}
\begin{itemize}
    \item \textbf{Resultados id\'enticos a Estrategia 1:} Mismo makespan, mismo desbalance extremo
    \item \textbf{M\'as nodos explorados:} Peor que Estrategia 1 en casi todas las instancias
    \begin{itemize}
        \item data01: 15.2M nodos (vs 9.4M en Estrategia 1)
        \item data05: 16.7M nodos (vs 8.5M en Estrategia 1)
    \end{itemize}
    \item \textbf{M\'as propagaciones:} Overhead adicional sin beneficio
\end{itemize}

\textbf{Conclusi\'on:} Decidir operarios primero sin informaci\'on temporal resulta en decisiones pobres que luego no pueden revertirse. Esta estrategia es \textbf{estrictamente inferior} a las otras dos.

\subsection{An\'alisis Comparativo}

\subsubsection{Comparaci\'on de Makespan}

\begin{figure}[h]
\centering
\begin{tabular}{lrrr}
\toprule
\textbf{Instancia} & \textbf{Estrategia 1} & \textbf{Estrategia 2} & \textbf{Estrategia 3} \\
\midrule
data00 & 86 & \textbf{39} & 86 \\
data01 & 108 & \textbf{57} & 108 \\
data02 & 112 & \textbf{54} & 112 \\
data03 & 120 & \textbf{45} & 120 \\
data04 & 118 & \textbf{62} & 118 \\
data05 & 120 & \textbf{43} & 120 \\
data06 & 138 & \textbf{65} & 138 \\
data07 & 162 & \textbf{86} & 162 \\
data08 & 155 & \textbf{66} & 155 \\
data09 & 175 & \textbf{71} & 175 \\
data10 & 187 & \textbf{84} & 187 \\
\midrule
\textbf{Mejora media} & --- & \textbf{54.3\%} & --- \\
\bottomrule
\end{tabular}
\caption{Comparaci\'on de makespan entre estrategias}
\label{tab:comp_makespan}
\end{figure}

\textbf{Conclusiones:}
\begin{itemize}
    \item Estrategia 2 \textbf{siempre} encuentra mejores soluciones
    \item Mejora promedio: 54.3\% reducci\'on de makespan
    \item Mejora m\'axima: 62.5\% (data03)
    \item Mejora m\'inima: 47.2\% (data01)
\end{itemize}

\subsubsection{Comparaci\'on de Balanceo}

El desbalance mide la diferencia entre la carga m\'axima y m\'inima de los operarios:

\begin{itemize}
    \item \textbf{Estrategias 1 y 3:} Desbalance = Makespan (un operario hace todo)
    \item \textbf{Estrategia 2:} Desbalance $\leq$ 6 en todas las instancias
    \begin{itemize}
        \item 3 instancias con balanceo perfecto (desbalance = 0)
        \item 6 instancias con desbalance = 1 (casi perfecto)
    \end{itemize}
\end{itemize}

\textbf{Ejemplo concreto (data03):}
\begin{itemize}
    \item Carga total: 120
    \item Operarios: 4
    \item Estrategia 2: carga = [30, 30, 30, 30] $\rightarrow$ desbalance = 0
    \item Estrategias 1/3: carga = [120, 0, 0, 0] $\rightarrow$ desbalance = 120
\end{itemize}

\subsubsection{Comparaci\'on de Eficiencia Computacional}

\begin{table}[h]
\centering
\small
\begin{tabular}{lrrr}
\toprule
\textbf{M\'etrica} & \textbf{Estrategia 1} & \textbf{Estrategia 2} & \textbf{Estrategia 3} \\
\midrule
Nodos promedio & 8,359,786 & 5,720,751 & 11,914,174 \\
Propagaciones promedio & 816,191,255 & 771,920,976 & 992,342,155 \\
Mejor instancia (nodos) & 7,500,955 & 4,061 & 8,088,673 \\
\bottomrule
\end{tabular}
\caption{Eficiencia computacional promedio}
\label{tab:eficiencia}
\end{table}

\textbf{Observaciones:}
\begin{itemize}
    \item Estrategia 2 tiene el \textbf{mejor promedio de nodos}
    \item Estrategia 3 es la \textbf{menos eficiente} (42\% m\'as nodos que Estrategia 1)
    \item La varianza en Estrategia 2 es alta: desde 4K hasta 11M nodos
    \item Estrategias 1 y 3 muestran comportamiento consistentemente malo
\end{itemize}

\subsubsection{An\'alisis de Escalabilidad}

Observando la tendencia de nodos explorados con el tama\~no del problema:

\begin{itemize}
    \item \textbf{Estrategias 1 y 3:} Nodos decrece ligeramente con instancias grandes
    \begin{itemize}
        \item Raz\'on probable: Alcanzan l\'imite de tiempo antes, reportan soluciones triviales tempranas
    \end{itemize}
    \item \textbf{Estrategia 2:} Comportamiento no mon\'otono
    \begin{itemize}
        \item data03, data05: Muy eficiente (estructura favorable)
        \item data00, data02, data06: Menos eficiente (estructura compleja)
    \end{itemize}
\end{itemize}

\textbf{Interpretaci\'on:} La estructura espec\'ifica de cada instancia (distribuci\'on de duraciones, n\'umero de operarios) afecta significativamente la dificultad del problema para Estrategia 2, mientras que las otras estrategias fallan uniformemente.

\subsection{Conclusiones del Problema 1.2a}

\begin{enumerate}
    \item \textbf{Importancia cr\'itica de la estrategia de b\'usqueda:} Los resultados demuestran que la elecci\'on de estrategia puede significar la diferencia entre soluciones triviales inaceptables y soluciones de alta calidad. En este problema, la diferencia no es marginal sino \textbf{cualitativa}.
    
    \item \textbf{Superioridad de dom\_w\_deg para problemas de scheduling multi-recurso:} La heur\'istica din\'amica \texttt{dom\_w\_deg} demostr\'o adaptarse efectivamente a la estructura del problema, logrando:
    \begin{itemize}
        \item Reducci\'on promedio del makespan del 54.3\%
        \item Balanceo casi perfecto de cargas (desbalance $\leq$ 6)
        \item Eficiencia computacional superior en promedio
    \end{itemize}
    
    \item \textbf{El orden de las decisiones importa:} Decidir asignaciones de operarios antes que tiempos (Estrategia 3) resulta en decisiones prematuras sin suficiente informaci\'on, llevando a soluciones pobres. El enfoque correcto es decidir tiempos primero (donde la informaci\'on de duraciones y precedencias guía la b\'usqueda) y luego asignar operarios.
    
    \item \textbf{Efectividad de restricciones redundantes:} La restricci\'on \texttt{cumulative}, aunque redundante l\'ogicamente, proporciona poda adicional crucial. Sin embargo, los resultados muestran que \textbf{por s\'i sola no es suficiente}: Estrategias 1 y 3 incluyen esta restricci\'on pero fallan igualmente.
    
    \item \textbf{Objetivos m\'ultiples requieren pesos cuidadosos:} La funci\'on $W \cdot \text{end} + \text{desbalance}$ con $W = \text{total} + 1$ efect\'ivamente prioriza makespan sin ignorar completamente el balanceo, pero \textbf{solo cuando la estrategia de b\'usqueda explora soluciones diversas}. Las estrategias por defecto convergen prematuramente a soluciones con un solo operario.
    
    \item \textbf{Ruptura de simetr\'ias esencial:} Las restricciones de ordenamiento de operarios (uso consecutivo, ordenamiento de cargas) son fundamentales para reducir el espacio de b\'usqueda. Sin embargo, deben combinarse con estrategias de b\'usqueda apropiadas para ser efectivas.
    
    \item \textbf{Variabilidad estructural:} Diferentes instancias con tama\~nos similares pueden tener dificultades muy distintas. La relaci\'on operarios/tareas, distribuci\'on de duraciones y estructura de precedencias interact\'uan de formas complejas que afectan la eficiencia del solver.
    
    \item \textbf{L\'imites pr\'acticos de tiempo:} El l\'imite de 60 segundos fue necesario para completar todas las pruebas, pero probablemente interrumpi\'o la b\'usqueda en algunas instancias. Estrategia 2 podr\'ia encontrar soluciones a\'un mejores con m\'as tiempo, mientras que Estrategias 1 y 3 probablemente no mejorar\'ian significativamente.
\end{enumerate}

\textbf{Recomendaci\'on final:} Para problemas de Job Shop con operarios limitados, se recomienda enfáticamente usar estrategias de b\'usqueda secuenciales que decidan tiempos de inicio con heur\'isticas din\'amicas (\texttt{dom\_w\_deg} o similar) antes de asignar operarios con \texttt{first\_fail}. Las estrategias por defecto o que priorizan asignaciones de recursos antes que decisiones temporales deben evitarse.
