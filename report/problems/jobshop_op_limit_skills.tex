\section{Problema 1.2b: Job Shop con Operarios Limitados y Habilidades}

\subsection{Descripci\'on del Problema}

Esta es una extensi\'on del problema de operarios limitados donde adem\'as se consideran habilidades espec\'ificas: cada operaci\'on requiere ciertas habilidades y cada operario posee un conjunto de habilidades. Una operaci\'on solo puede ser asignada a un operario que posea todas las habilidades necesarias.

\textbf{Contexto industrial:} En entornos de producci\'on reales, no todos los operarios est\'an calificados para todas las tareas. Las certificaciones, experiencia y especializaciones determinan qui\'en puede operar cada m\'aquina o realizar cada operaci\'on.

\textbf{Elementos adicionales:}
\begin{itemize}
    \item Cada operaci\'on requiere un conjunto de habilidades espec\'ificas
    \item Cada operario posee un conjunto de habilidades
    \item Un operario solo puede ejecutar operaciones para las cuales tiene todas las habilidades requeridas
    \item Sigue habiendo limitaci\'on en el n\'umero de operarios (k < n\'umero de m\'aquinas)
    \item \textbf{Objetivo dual:} Minimizar makespan y balancear la carga de trabajo
\end{itemize}

\subsection{Modelamiento como CSP}

\subsubsection{Par\'ametros del Modelo}

[Por completar con la descripci\'on de par\'ametros]

\subsubsection{Variables de Decisi\'on}

[Por completar con la descripci\'on de variables]

\subsubsection{Dominios}

[Por completar con la descripci\'on de dominios]

\subsubsection{Restricciones}

[Por completar con la descripci\'on de restricciones principales]

\subsection{Detalles de Implementaci\'on en MiniZinc}

\subsubsection{Aspectos Relevantes}

[Por completar con aspectos t\'ecnicos de la implementaci\'on]

\subsubsection{Restricciones Redundantes}

[Por completar con restricciones redundantes y su justificaci\'on]

\subsubsection{Ruptura de Simetr\'ias}

[Por completar con estrategias de ruptura de simetr\'ias]

\subsection{Estrategias de B\'usqueda}

[Por completar con descripci\'on de estrategias exploradas]

\subsection{Pruebas Realizadas}

[Por completar con casos de prueba y resultados]

\subsection{An\'alisis de Resultados}

[Por completar con an\'alisis comparativo]

\subsection{Conclusiones del Problema 1.2b}

[Por completar con conclusiones espec\'ificas]
